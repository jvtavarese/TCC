A predição de qualidade de vida usando dados comportamentais de \textit{wearables} tem mostrado resultados promissores em estudos anteriores, com RMSE entre 6–7 pontos. Este trabalho investiga se tais resultados refletem a capacidade real de generalização ou artefatos metodológicos. Para isso, o \textit{dataset} Healful (1.373 registros originais, 35 participantes) é analisado e aplicado o tratamento de \textit{outliers} e a imputação de valores faltantes (\textit{forward fill} e interpolação), expandindo o dataset para 2.267 registros. Em seguida,  cinco cenários experimentais são comparados: (A) metodologia original com \textit{KFold} e \textit{features Featurewiz} (40), (B) metodologia original com \textit{dataset} pós-VIF (60 \textit{features}), (C) \textit{GroupKFold} sem \textit{data leakage} com \textit{features} originais, (D) \textit{GroupKFold} com \textit{dataset} processado e modelos tradicionais, e (E) \textit{GroupKFold} com modelos avançados (XGBoost, LightGBM, CatBoost). Os resultados demonstram que: (1) a remoção de multicolinearidade via VIF proporciona melhoria de 27–29\% no RMSE (cenários A→B: 9,06→6,43 \textit{Physical}, 9,23→6,73 \textit{Psychological}); (2) a aplicação de \textit{GroupKFold} para eliminar \textit{data leakage} revela degradação crítica de 132–143\% no RMSE (A→D: 9,06→20,99 \textit{Physical}, 9,23→22,43 \textit{Psychological}); (3) o coeficiente de determinação $R^2$ torna-se consistentemente negativo em todos os cenários rigorosos ($-0,37$ a $-2,14$), indicando que os modelos erram mais que o \textit{baseline} de prever a média; (4) modelos avançados melhoram a performance em 16,8\% (\textit{Physical}) e 7,5\% (\textit{Psychological}) em comparação aos modelos tradicionais, porém o $R^2$ permanece negativo. A decomposição de variância revelou que 83\% da variabilidade na qualidade de vida é individual (\textit{between-participants}) e apenas 12\% temporal (\textit{within-participants}), explicando a dificuldade de generalização. Essa desproporção, combinada com o tamanho amostral limitado (35 participantes, aproximadamente 7 por \textit{fold} de teste), torna inviável a generalização para novos indivíduos. Conclui-se que modelos populacionais apresentam utilidade limitada para a predição de qualidade de vida em novos participantes, sendo necessárias abordagens personalizadas ou \textit{datasets} significativamente maiores (mais de 100 participantes). Este estudo contribui metodologicamente ao demonstrar a importância de validação rigorosa com \textit{GroupKFold}, quantificar o impacto do \textit{data leakage} (132–143\%) e evidenciar o valor do pré-processamento por meio de VIF (27–29\% de melhora).

% Separe as palavras-chave por ponto e vírgula (norma UFC)
\palavraschave{qualidade de vida; dispositivos vestíveis; aprendizado de máquina; validação cruzada; vazamento de dados; heterogeneidade individual; GroupKFold; VIF.}