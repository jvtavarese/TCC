
A Deus, por todas as bênçãos concedidas e pela força nos momentos de desafio. Com fé, pude percorrer este caminho e chegar até aqui.

À minha família, pelo apoio incondicional e por compreenderem os momentos de ausência dedicados aos estudos. O incentivo constante e a confiança depositada em mim foram fundamentais para esta conquista.

À minha namorada, que acompanhou toda a minha trajetória acadêmica desde o início, sempre incentivando meus sonhos e me motivando a buscar mais. Seu apoio foi essencial durante esta jornada.

À Professora Dra. Rossana Maria de Castro Andrade, que prontamente aceitou o desafio de me orientar neste trabalho e conduziu a orientação de forma excepcional, contribuindo significativamente para o desenvolvimento desta pesquisa.

Aos amigos da graduação, que tornaram esta caminhada mais leve e compartilharam os desafios e conquistas ao longo do curso. Sem vocês, certamente teria sido muito mais difícil.

Ao Professor Dr. Pedro Almir Martins de Oliveira, pelos esclarecimentos valiosos e pela disponibilidade em contribuir com este trabalho, elevando sua qualidade técnica e científica.

A todos os professores do curso de Engenharia de Computação da Universidade Federal do Ceará, pelos ensinamentos não apenas técnicos, mas também éticos e profissionais, que levarei para toda a vida.

Ao Doutorando em Engenharia Elétrica, Ednardo Moreira Rodrigues, e a Alan Batista de Oliveira pela adequação do \textit{template} utilizado neste trabalho para que o mesmo ficasse de acordo com as normas da biblioteca da Universidade Federal do Ceará (UFC).

Aos bibliotecários da Universidade Federal do Ceará: Francisco Edvander Pires Santos, Juliana Soares Lima, Izabel Lima dos Santos, Kalline Yasmin Soares Feitosa e Eliene Maria Vieira de Moura, pela revisão e contribuição para a formatação deste trabalho.
