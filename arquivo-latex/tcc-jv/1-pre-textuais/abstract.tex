Quality of life prediction using behavioral data from wearables has shown promising results in previous studies, with RMSE between 6–7 points. This work investigates whether such results reflect true generalization capability or methodological artifacts. We analyzed the Healful dataset (1,373 original records, 35 participants) applying outlier treatment and missing value imputation (forward fill and interpolation), expanding to 2,267 records. We compared five experimental scenarios: (A) original methodology with KFold and Featurewiz features (40), (B) original methodology with post-VIF dataset (60 features), (C) GroupKFold without data leakage with original features, (D) GroupKFold with processed dataset and traditional models, and (E) GroupKFold with advanced models (XGBoost, LightGBM, CatBoost). The results demonstrate that: (1) multicollinearity removal via VIF provides 27–29\% improvement in RMSE (scenarios A→B: 9.06→6.43 Physical, 9.23→6.73 Psychological); (2) GroupKFold application to eliminate data leakage reveals critical degradation of 132–143\% in RMSE (A→D: 9.06→20.99 Physical, 9.23→22.43 Psychological); (3) the coefficient of determination $R^2$ becomes consistently negative in all rigorous scenarios ($-0.37$ to $-2.14$), indicating that models perform worse than the baseline of predicting the mean; (4) advanced models improve performance by 16.8\% (Physical) and 7.5\% (Psychological) compared to traditional models, but $R^2$ remains negative. Variance decomposition revealed that 83\% of quality of life variability is individual (between-participants) and only 12\% temporal (within-participants), explaining the generalization difficulty. This disproportion, combined with limited sample size (35 participants, approximately 7 per test fold), makes generalization to new individuals unfeasible. We conclude that population models have limited utility for quality of life prediction in new participants, requiring personalized approaches or significantly larger datasets (over 100 participants). This study contributes methodologically by demonstrating the importance of rigorous validation with GroupKFold, quantifying the impact of data leakage (132–143\%), and evidencing the value of preprocessing through VIF (27–29\% improvement).

% Separate keywords by semicolon (UFC standard)
\keywords{quality of life; wearables; machine learning; cross-validation; data leakage; individual heterogeneity; GroupKFold; VIF.}
